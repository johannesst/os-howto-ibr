\documentclass[a4paper,ngerman,bibtotocliststotoc]{scrartcl}
\usepackage[automark]{scrpage2}
\pagestyle{scrheadings}
\usepackage{multicol}
%\usepackage{multirow}
\usepackage{graphicx}
%\usepackage{caption}
%\usepackage{comment}
%\usepackage{mathdots}
%\usepackage{natbib}
%\usepackage[utf8x]{inputenc}
%\usepackage
\renewcommand*{\titlepagestyle}{empty}
%Mehr Abstand bei Fußnoten
%Hack von Markus Kohm. Gefunden auf:
%<http://groups.google.de/group/de.comp.text.tex/msg/6ab39a8c01a6c06b>
%\newcommand*{\OrigFootnoteRule}{}
%\let\OrigFootnoteRule=\footnoterule
%\newcommand*{\semiflushbottom}{%
%  \raggedbottom%
%    \renewcommand\footnoterule{%
%        \vskip 0pt plus .001fil%
%	    \OrigFootnoteRule}
%
%	    } 
%Ein anderer Hack von Patrick Pommé scheint zu tun
%<http://groups.google.de/group/de.comp.text.tex/msg/a00556927050e1b1>
\renewcommand{\footnoterule}{\vspace{0,25cm} \rule{5cm}{0.4pt}
\vspace{0,25cm}} 
%\usepackage[bottom]{footmisc}
\usepackage{comment}
 \usepackage{lmodern}
% ersetzen
% Was soll das bedeuten? Es hat was mit der Encodierung zu tun, aber
% was? Kann mich jemand aufklären?
\usepackage[T1]{fontenc}
% Zeichensatz einstellen
\usepackage[utf8]{inputenc}
% Paket für mehrere Sprachen in einen Dokument einbinden
% Wird für die "neue deutsche Rechtschreibung" benötigt
\usepackage{babel}
% Pakete für mathematischen Formelsatz einbinden
\usepackage{amsmath}
\usepackage{amssymb}
% Typographie verbessern
\usepackage{microtype}
\usepackage[
hyperfootnotes=true,
bookmarks,
bookmarksopen=true,
bookmarksnumbered=true,
colorlinks={false},
urlcolor={blue},
linkcolor={blue},
citecolor={blue},
pdfstartview={FitH},
pdfauthor={Johannes Starosta <j.starosta@tu-bs.de>}]{hyperref}
%\newcommand*{\teil}[1]{\section*{\setcounter{part}{1} \Huge Teil 
 %   \thepart :  #1}}
\begin{document}%\
      %\subject{TU Braunschweig\\IBR/Distributed Systems}
      \title{Configuring and using the openstack cloud at IBR}
\author{Johannes Starosta <j.starosta@tu-bs.de>}
\date{September 2012}
%\titlehead{
%    \centering
%\includegraphics{logo_tu}\\
%}
\maketitle
\tableofcontents
%\newpage
%\begin{multicols}{1}
%%\section*{Vorbemerkung}
%\url{J.Starosta@tu-bs.de}.
%\tableofcontents 
%\end{multicols}
\begin{abstract}
This document describes how you use and administrate the openstack cloud
of the Distributed Systems Group at TU Braunschweig. It also describes
some basic concepts of openstack. \emph{Warning: It's not a replace of
the official documentation, use with caution!}
\end{abstract}
%struktur:
%einführung komponenten, welche clients
%Benutzermanagment-Konzepte und Handhabung
%Zugriff via nova,euca-tools, hyperfox,dashboard
%Image Erzeugung
\section{Introduction}
\label{sec:introduction}
Openstack consist of five modules, which serves different purposes:
\begin{itemize}
\item The Open Stack Compute Infrastructure is provided by the
  nova-daemon. nova is responsible for:
  \begin{itemize}
  \item  Instance life cycle management
  \item Management of compute resources
  \item Networking and Authorization
  \end{itemize}
It also provides a web service API, which can be used to control the
instances. It's compatible with the EC2 API of Amazon Web Services and
can be used by:
\begin{itemize}
\item The nova-pythonclient, the euca2ools or any other EC2 compatible
  client, the Horizon Webinterface of Openstack
\end{itemize}
\item The OpenStack Imaging Service  ist provided by the glance
  daemon. It's used for retrival, lookup and uploading of virtual
  machine images. It can be accessed via the glance CLI client, any
  EC2/S3 compatible client and Horizon
\item OpenStack Object Storage is provided by the Swift daemon. It can
  be acced by the swift CLI tool, Horizizon oder any Amazon S3 compatible client
\item OpenStack Identity Service  (Keystone) is used to set up user
  authentication and authorization. It can be used by using the
  keystone CLI tool.
\end{itemize}
\section{Configuring users and their permissions with the Openstack
  Identity Service (Keystone)}

\subsection{Basic concepts: Users, tenants and roles}
\label{sec:basic-conc-users}
There are three main concepts of Identity user management are in Openstack:
\begin{itemize}
\item Users: Represents a human user (e.g. Alice,Bob) and has associated
  information such as usernae, password and email
\item Tenants: A tenant can be thought of as a project, group, or
  organization (e.G: group1,IBR...) 
\item Roles: A role captures what operations a user is permitted to
  perform in a given tenant.  A user can have different roles in
  different tenants. Here is an example: Imagine the users Alice and
  Bob. Both are members of the tenant IBR, as well as of the tenant
  group1. Now see their roles:\\
  \begin{center}
  \begin{tabular}{c|c|c}
    User & Tenant & Role  \tabularnewline \hline
    Alice &IBR& admin \tabularnewline \hline
    Bob &IBR&Member  \tabularnewline \hline
    Alice&group1&Member \tabularnewline \hline
    Bob&group1&Member
  \end{tabular}  
  \end{center}
  Alice has the role ,,admin'' of the  tenant ,,IBR'' while
  Bob has the role ,,users''. In the tenant ,,group1'' Alice and Bob
  are have the role ''users''. Please note, that ,,admin'' and
  ,,users' are just names, unless you define the permission polices of a
  certain role. The currenty installed police on the Openstack Cloud
  at IBR just defines the roles ,,admin'' and ,,swiftoperator''.
  A user need to have the role ,,admin'' to do some administrative tasks (e.G: Creating
  users...), which should not be allowed for normal users. If a user
  should be given access to the swift object storage service, he
  should get the role ,,swiftoperator''.\\
Note: A user can have several roles for the same tenant:
\begin{center}
  \begin{tabular}{c|c|c}
    User & Tenant & Role  \tabularnewline \hline
    Bob &IBR&Member \tabularnewline \hline
    Bob & IBR&swiftoperator%\tabularnewline 
  \end{tabular}  
  \end{center}
 For further information see %TODO: chapter 6, seite 76 of OS compute
                             %administration essex
                             % \item 
\end{itemize}

\subsection{Creating and managing users, tenants and roles}
\label{sec:creat-manag-users}
Now we want to create an user. Given is following scenario:\\
Alice and Bob are students of Eve. Alice and Bob participates in Eves
cloud computing lab. So Eve wants them to work together on a shared
project. They should be able to create custom images, upload them and
run their own cloud instances (thus they need access to compute and
glance). They also should use Swift Object Storage, so they can upload
their documentation to an Object Storage container. Eve decides to
create a tenant ,,cclab-group1'' for them. Eve wants to have acces to
the tenants instances as well. She also wants to 
have access to the
tenant as well and she wants, administrative permissions, while Alice and Bob shall be
just normal users... So she want to create following
setup:

\begin{center}
  \begin{tabular}{ccc}
    User & Tenant & Role\\
    Alice&cclab-group1&Member\\
    Bob  &cclab-group1&Member\\
    Alice&cclab-group1&swiftoperator\\
    Bob  &cclab-group1&swiftoperator\\
    Eve  &cclab-group1&admin\\
  \end{tabular}
\end{center}
\fbox{\parbox{\textwidth
  }{\begin{center}\LARGE{\textbf{WARNING:}}\end{center}
\emph{The environment on
  cloud2 is configured to do administrative tasks. Don't use it to do
  things, you want to use with a certain user! For example if you
  create an virtual machine instance as user ,,admin'', you won't be
  able to access it from your normal user account. In %TODO reference
                                %to usage chapter
  we'll discuss how to access your user environment without messing up
  with the admin account.}}}% \hfill

She connects to \verb|cloud2.ibr.cs.tu-bs.de| and run following
commands in the admin console\footnote{These are NOT real data
  ids. Don't use them in the real world!}
\begin{verbatim}
# Every line, which starts with a # is a comment
# First she wants to know, which users, roles and tenants alreday
#exists
eve@uncinus:~$ keystone user-list
+----------------------------------+---------+-------------------------+-----------+
|                id                | enabled |          email          |    name   |
+----------------------------------+---------+-------------------------+-----------+
| 1f53b11a2e0e40239b25961380472c54 |   True  |  admin@ibr.cs.tu-bs.de  |   swift   |
| 552bfa0c2d6847b6a8ea2625bff5531f |   True  |  admin@ibr.cs.tu-bs.de  |   glance  |
| cfb1870fc4c84a56bd1f19a5e577d984 |   True  |  admin@ibr.cs.tu-bs.de  |    nova   |
| dd69242cbdb243dd803b7641c223adb4 |   True  |  admin@ibr.cs.tu-bs.de  |   admin   |
| eveIDint                         |   True  |   eve@ibr.cs.tu-bs.de   |    eve    |
+----------------------------------+---------+-------------------------+-----------+
eve@uncinus:~$
eve@uncinus:~$ keystone tenant-list
+----------------------------------+----------+---------+
|                id                |   name   | enabled |
+----------------------------------+----------+---------+
| 0b6336de185a4f01a59fcb5ef9f40d96 | service  |   True  |
| 9f4b31c709b2431b972666100cf12c79 |  users   |   True  |
| fa188ad427e24bb6b14b4945fc6af6da |  admin   |   True  |
+----------------------------------+----------+---------+
eve@uncinus:~$ keystone role-list
+----------------------------------+---------------+
|                id                |      name     |
+----------------------------------+---------------+
| 2a292fb8580e49c095059a7b258b57f3 | swiftoperator |
| 45649d6a8b044f1d860905a896b3473b |     Member    |
| 47870520359d4f6b9d5d87a175a2f1f1 |     admin     |
+----------------------------------+---------------+
eve@uncinus:~$
# We already have all needed roles, eve still need to create the
# users, tenant and assign them to the needed roles
eve@uncinus:~$ keystone user-create --name bob --email bob@ibr.cs.tu-bs.de --pass bobpass
+----------+-------------------------------------------------------------------------------------------------------------------------+
| Property |                                                          Value                                                          |
+----------+-------------------------------------------------------------------------------------------------------------------------+
|  email   |                                                   bob@ibr.cs.tu-bs.de                                                   |
| enabled  |                                                           True                                                          |
|    id    |                                                        bobsIDint                                                           |
|   name   |                                                           bob                                                           |
| password | $6$rounds=40000$0p.sAOcBsq8ms/Sk$B/yUdMQoxAwpM8kB1RwyHuL.mYrLI/BZZxlrDeT1U.Rm6QfmBn5/MA86KcBHz90Hr67bOV8EGdfUbOkJ0JwBW. |
| tenantId |                                                           None                                                          |
+----------+-------------------------------------------------------------------------------------------------------------------------+
eve@uncinus:~$ keystone user-create --name alice --email alice@ibr.cs.tu-bs.de --pass alicepass
+----------+-------------------------------------------------------------------------------------------------------------------------+
| Property |                                                          Value                                                          |
+----------+-------------------------------------------------------------------------------------------------------------------------+
|  email   |                                                 alice@ibr.cs.tu-bs.de                                                   |
| enabled  |                                                           True                                                          |
|    id    |                                                        aliceIDint                                                       |
|   name   |                                                         alice                                                           |
| password | $6$rounds=40000$0p.sAOcBsq8ms/Sk$B/yUdMQoxAwpM8kB1RwyHuL.mYrLI/BZZxlrDeT1U.Rm6QfmBn5/MA86KcBHz90Hr67bOV8EGdfUbOkJ0JwBW. |
| tenantId |                                                           None                                                          |
+----------+-------------------------------------------------------------------------------------------------------------------------+
#now eve creates the tenant
eve@uncinus:~$ keystone tenant-create  --name cclab-group1
+-------------+----------------------------------+
|   Property  |              Value               |
+-------------+----------------------------------+
| description |               None               |
|   enabled   |               True               |
|      id     | 898daab4fc4d45b98f231872241458e4 |
|     name    |           cclab-group1           |
+-------------+----------------------------------+
eve@uncinus:~$ 
# at the end we assign the needed roles to every user
#eve got role admin
eve@uncinus:~$ keystone user-role-add --user eveIDint  --role 47870520359d4f6b9d5d87a175a2f1f1 --tenant_id 898daab4fc4d45b98f231872241458e4
eve@uncinus:~$
#bob and alice got the role Member
eve@uncinus:~$ keystone user-role-add --user bobIDint    --role 45649d6a8b044f1d860905a896b3473b --tenant_id 898daab4fc4d45b98f231872241458e4
eve@uncinus:~$ keystone user-role-add --user aliceIDint  --role 45649d6a8b044f1d860905a896b3473b --tenant_id 898daab4fc4d45b98f231872241458e4
#bob and alice got the role swiftoperator
eve@uncinus:~$ keystone user-role-add --user bobIDint    --role 2a292fb8580e49c095059a7b258b57f3 --tenant_id 898daab4fc4d45b98f231872241458e4
eve@uncinus:~$ keystone user-role-add --user aliceIDint  --role 2a292fb8580e49c095059a7b258b57f3 --tenant_id 898daab4fc4d45b98f231872241458e4
\end{verbatim}
Now everything ist configured. Of course Eve could have add additional
roles or more tenants. More information how to do identify managment
can be found in %For background information refer to
% FIXME referenz einfügen.

\subsection{Updating user information}
\label{sec:updat-user-inform}
You might want to change details of the users. Again we look in our
scenario:
Bob complains, that he can't remember his password. He prefers to have
the password "swordfish". He also would prefered to have another email
(bob@tu-bs.de) in the user-details.
Again Eve connects to \verb|clou2.ibr.cs.tu-bs.de|:
\begin{verbatim}
eve@uncinus:~$
eve@uncinus:~$ keystone user-update --email bob@tu-bs.de bobIDint
eve@uncinus:~$ keystone user-password-update --pass swordfish bobIDint
\end{verbatim}
You can change the details of names and roles as well. Please refer to
% FIXME

\subsection{Removing everything}
\label{sec:removing-everything}
Now the term is finished. Alice and Bob have finished their project and
passed the examination. Eve now wants to removes their user-account,
tenant, since the Cloud will be needed for new students next term:
\begin{verbatim}
#first eve remove the roles from the tenant
eve@uncinus:~$ keystone user-role-remove --user eveIDint    --role 47870520359d4f6b9d5d87a175a2f1f1 --tenant_id 898daab4fc4d45b98f231872241458e4
eve@uncinus:~$ keystone user-role-remove --user bobIDint    --role 45649d6a8b044f1d860905a896b3473b --tenant_id 898daab4fc4d45b98f231872241458e4
eve@uncinus:~$ keystone user-role-remove --user aliceIDint  --role 45649d6a8b044f1d860905a896b3473b --tenant_id 898daab4fc4d45b98f231872241458e4
eve@uncinus:~$ keystone user-role-remove --user bobIDint    --role 2a292fb8580e49c095059a7b258b57f3 --tenant_id 898daab4fc4d45b98f231872241458e4
eve@uncinus:~$ keystone user-role-remove --user aliceIDint  --role 2a292fb8580e49c095059a7b258b57f3 --tenant_id 898daab4fc4d45b98f231872241458e4
# now we remove the tenant:
eve@uncinus:~$ keystone tenant-delete  898daab4fc4d45b98f231872241458e4
# and in the end the accounts of alice and bob
eve@uncinus:~$ keystone user-delete aliceIDint
eve@uncinus:~  keystone user-remove bobIDint
\end{verbatim}
%\newpage
%\begin{multicols}{2}

\end{document}